% !TeX root = ../dd.tex
\section{Scope}
The CodeKataBattle platform is a distributed software system that allows its users (students) to participate in challenges of different programming languages to improve their software development skills.
Educators can create and run programming battles, and all students enrolled in the platform can join in.
Students can create teams and carry out the battles.
When a team loads a possible solution to the battle, the system runs tests for the battle and calculates the corresponding score.
\\\\
Implementation choices both abstractly and in detail are covered in later sections of this document in order to enable software implementation of the system.

\section{Definitions, Acronyms, Abbreviations}

\subsection{Definitions}
\begin{description}[leftmargin=0pt]
    \item[Slug:] a human-readable and URL-friendly (as in, limited to lowercase ASCII characters) string which
          identifies a particular resource. It is tipically used to identify resources in URLs, but can also be used
          as IDs instead of a more traditional way such as integers
    \item[GitHub Repository Slug:] a slug with the structure `<repository owner>/<repository name>' which uniquely
          identifies a repository across all of the ones hosted on GitHub. Can tipically be seen in a repository URL
\end{description}

\subsection{Acronyms}
\begin{description}[leftmargin=0pt]
    \item[CKB:] CodeKataBattle
    \item[API:] Application Programming Interface
    \item[SAT:] Static Analysis tools
    \item[GH:] GitHub
    \item[SSO:] Single Sign On
    \item[UUID:] Universal Unique Identifier
    \item[DB:] DataBase
    \item[DBMS:] DataBase Management System
    \item[RPC:] Remote Procedure Call
    \item[REST:] REpresentational State Transfer
    \item[SPA:] Single Page App
    \item[CDN:] Content Delivery Network
\end{description}

\subsection{Abbreviations}
\begin{description}[leftmargin=0pt]
    \item[e.g.:] For example
    \item[repo:] Repository
    \item[ID:] Identifier
\end{description}

\section{Revision history}

\begin{itemize}
    \item \textbf{1.0} {-} Initial release ()
\end{itemize}

\section{Reference Documents}

\begin{description}[leftmargin=0pt]
    \item[Specification document:] \emph{"Assignment RDD AY 2023-2024"}
    \item[UML official specification:] \url{https://www.omg.org/spec/UML/}
    \item[Requirements Analysis and Specification Document:] \emph{"RASD"}
\end{description}


\section{Document Structure}

\begin{enumerate}
    \item \textbf{Section 1: Introduction} \\
    \item \textbf{Section 2: Architectural Design} \\
    \item \textbf{Section 3: User Interface Design} \\
    \item \textbf{Section 4: Requirements Traceability} \\
    \item \textbf{Section 4: Implementation, Integration and Test Plan} \\
\end{enumerate}