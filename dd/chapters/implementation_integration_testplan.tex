% !TeX root = ../dd.tex

The choosen architecture guarantees high decoupling between components, so that most of them
can be developed and unit tested completely independently, without the need to mock other
components. Once those are developed, we can integrate them at the end and do additional
integration testing. We can identify a few different types components:

\begin{description}[leftmargin=0pt]
    \item[Independent components:] these are back-end components that can be developed fully independently
          from one another and do not need to directly integrate with any other service
    \item[External components:] these are the components provided by third-parties, which are supposed
          to be already reliable, but may need to be stubbed when unit testing our own services
    \item[Integrating components:] these are the back-end components whose sole role is of allowing
          communication between other services and therefore need to be directly integrated with those
    \item[Front-end components:] these are the presentational components that belong to the client layer
          which rely on the back-end REST API, they can therefore be unit tested by mocking it
\end{description}

In order to visualize how much each component will need to be tested and plan accordingly, we can use
the following table, where we associate to each piece of functionality the difficulty of its
implementation and its importance for the final user experience:

\begin{table}[H]
    %\caption*{\textbf{Title}}
    \centering
    \begin{tabular}{|l|l|l|}
        \hline
        \textbf{Feature}                    & \textbf{Importance} & \textbf{Difficulty} \\\hline
        Login                               & High                & Medium              \\
        Tournament and Battle Management    & High                & Low                 \\
        Tournament and Battle Participation & High                & Medium              \\
        Automatic evaluation                & High                & High                \\
        Manual evaluation                   & High                & Low                 \\
        Tournament rankings                 & Medium              & Low                 \\
        Battle rankings                     & Medium              & Low                 \\
        Notifications                       & Medium              & High                \\
        Badges Management                   & Low                 & Medium              \\
        Badges Assignment                   & Low                 & High                \\
        Badges Visualization                & Low                 & High                \\\hline
    \end{tabular}
    \caption{Importance and difficulty of features}
    \label{table:Importance and difficulty of features}
\end{table}

\subsection{Development and Test Plan}
All the components will be developed and tested using a bottom-up approach, in order to reduce as
much as possible stubs and mocks, which would add additional overhead to the development.
All the indipendent microservices can be developed first and in parallel, prioritizing components
of high importance which need to be tested more thoroughly as outlined by the table above.
Development can also start on front-end components, which can mock the REST API during testing.
Lastly, integrating components such as the API Gateway can be developed, after which everything
can be integration tested. After that is completed, adherence to the specified requirements
needs to be verified.

\subsection{Components integration}
Here components and subsystems are illustrated via graphs. \\
Subsystems are a group of components meant to be integration tested together after unit testing.\\
Integration testing is carried out in a bottom-up approach, so drivers must be implemented to test the different independent subsystems as each subsystem is developed.\\
The first components that can be tested together are the Notification Service and the RabbitMQ (third party service).
That's because the subsystem is implementing High Importance features and thus proper testing with the queue needs to be done carefully.
\begin{adjustbox}{
        max size={\textwidth}{\textheightwithcaption{1}},
        caption={Notitification subsystem},
        label={fig:Notitification subsystem},
        figure=H}
    \centering
    \puml{puml/components-integration/notification}
\end{adjustbox}

Then the Platform subsystem can be integreted with the Queue of the Notification subsystem.
Obtaining the Platform and Notification Subsystem.
\begin{adjustbox}{
        max size={\textwidth}{\textheightwithcaption{1}},
        caption={Platform and Notification subsystem},
        label={fig:Platform and Notification subsystem},
        figure=H}
    \centering
    \puml{puml/components-integration/platform-queue}
\end{adjustbox}

The following four subsystems might be developed in any order, or simultaneously, as they do not depend
on each other.
It is recommended to develop first the subsystems that implement High Importance features, such as:\\
the User subsystem, the Build and Test subsystem, and the Static Analysis subsystem, and then the Badge subsystem that implements Low Importance functionality.
\begin{adjustbox}{
        max size={\textwidth}{\textheightwithcaption{1}},
        caption={User subsystem},
        label={fig:User subsystem},
        figure=H}
    \centering
    \puml{puml/components-integration/user}
\end{adjustbox}

\begin{adjustbox}{
        max size={\textwidth}{\textheightwithcaption{1}},
        caption={Build and Test subsystem},
        label={fig:Build and Test subsystem},
        figure=H}
    \centering
    \puml{puml/components-integration/buildtest}
\end{adjustbox}

\begin{adjustbox}{
        max size={\textwidth}{\textheightwithcaption{1}},
        caption={Static Analysis subsystem},
        label={fig:Static Analysis  subsystem},
        figure=H}
    \centering
    \puml{puml/components-integration/staticanalysis}
\end{adjustbox}

\begin{adjustbox}{
        max size={\textwidth}{\textheightwithcaption{1}},
        caption={Badge subsystem},
        label={fig:Badge subsystem},
        figure=H}
    \centering
    \puml{puml/components-integration/badge}
\end{adjustbox}

After that we can also develop the Website CDN.
\begin{adjustbox}{
        max size={\textwidth}{\textheightwithcaption{1}},
        caption={Website CDN subsystem},
        label={fig:Website CDN subsystem},
        figure=H}
    \centering
    \puml{puml/components-integration/website}
\end{adjustbox}

At this point all subsystems have been implemented and tested, so the next step is performing integration
testing between the subsystems and the API Gateway.
\begin{adjustbox}{
        max size={\textwidth}{\textheightwithcaption{1}},
        caption={API Gateway subsystem},
        label={fig:API Gateway subsystem},
        figure=H}
    \centering
    \puml{puml/components-integration/apigateway}
\end{adjustbox}

The integration of all the different subsystems with the Gateway API is a critical operation and performing it all at once will likely lead to errors (because of the big-bang like approach).
To avoid this, the integration can be broken down into smaller integrations by continuing to follow the bottom-up approach.\\
We can proceed to integrate the API Gateway with the Platform and Notification subsystem.

\begin{adjustbox}{
        max size={\textwidth}{\textheightwithcaption{1}},
        caption={API Gateway subsystem step 1},
        label={fig:API Gateway subsystem step 1},
        figure=H}
    \centering
    \puml{puml/components-integration/api_integration1}
\end{adjustbox}

Then, it's possible to integrate the API Gateway with the User and the Badge subsystems.


\begin{adjustbox}{
        max size={\textwidth}{\textheightwithcaption{1}},
        caption={API Gateway subsystem step 2},
        label={fig:API Gateway subsystem step 2},
        figure=H}
    \centering
    \puml{puml/components-integration/api_integration2}
\end{adjustbox}

After that, integrate the API Gateway with the Static Analysis and Build and Test subsystems.

\begin{adjustbox}{
        max size={\textwidth}{\textheightwithcaption{1}},
        caption={API Gateway subsystem step 3},
        label={fig:API Gateway subsystem step 3},
        figure=H}
    \centering
    \puml{puml/components-integration/api_integration3}
\end{adjustbox}

Finally, when all is integrated and tested sucessfully we can proceed to integrate it with the Website CDN.

\begin{adjustbox}{
        max size={\textwidth}{\textheightwithcaption{1}},
        caption={API Gateway subsystem step 4},
        label={fig:API Gateway subsystem step 4},
        figure=H}
    \centering
    \puml{puml/components-integration/api_integration4}
\end{adjustbox}