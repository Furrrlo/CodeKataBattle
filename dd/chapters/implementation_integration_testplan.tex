% !TeX root = ../dd.tex

The choosen architecture guarantees high decoupling between components, so that most of them
can be developed and unit tested completely independently, without the need to mock other
components. Once those are developed, we can integrate them at the end and do additional
integration testing. We can identify a few different types components:

\begin{description}[leftmargin=0pt]
     \item[Independent components:] these are back-end components that can be developed fully independently
           from one another and do not need to directly integrate with any other service
     \item[External components:] these are the components provided by third-parties, which are supposed
           to be already reliable, but may need to be stubbed when unit testing our own services
     \item[Integrating components:] these are the back-end components whose sole role is of allowing
           communication between other services and therefore need to be directly integrated with those
     \item[Front-end components:] these are the presentational components that belong to the client layer
           which rely on the back-end REST API, they can therefore be unit tested by mocking it
\end{description}

In order to visualize how much each component will need to be tested and plan accordingly, we can use
the following table, where we associate to each piece of functionality the difficulty of its 
implementation and its importance for the final user experience:

\begin{table}[H]
     %\caption*{\textbf{Title}}
     \centering
     \begin{tabular}{|l|l|l|}
          \hline
          \textbf{Feature}                    & \textbf{Importance} & \textbf{Difficulty} \\\hline
          Login                               & High                & Medium              \\
          Tournament and Battle Management    & High                & Low                 \\
          Tournament and Battle Participation & High                & Medium              \\
          Automatic evaluation                & High                & High                \\
          Manual evaluation                   & High                & Low                 \\
          Tournament rankings                 & Medium              & Low                 \\
          Battle rankings                     & Medium              & Low                 \\
          Notifications                       & Medium              & High                \\
          Badges Management                   & Low                 & Medium              \\
          Badges Assignment                   & Low                 & High                \\
          Badges Visualization                & Low                 & High                \\\hline
     \end{tabular}
     \caption{Importance and difficulty of features}
     \label{table:Importance and difficulty of features}
\end{table}

\subsection{Development and Test Plan}
All the components will be developed and tested using a bottom-up approach, in order to reduce as
much as possible stubs and mocks, which would add additional overhead to the development.
All the indipendent microservices can be developed first and in parallel, prioritizing components 
of high importance which need to be tested more thoroughly as outlined by the table above.
Development can also start on front-end components, which can mock the REST API during testing.
Lastly, integrating components such as the API Gateway can be developed, after which everything
can be integration tested. After that is completed, adherence to the specified requirements
needs to be verified.

\subsection{Components integration}

