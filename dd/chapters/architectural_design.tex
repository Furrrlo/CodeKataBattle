% !TeX root = ../dd.tex
\section{Overview}
The system is a distributed application which follows the microservices architecture paradigm.\\
With the term client we are referring to either a Student or an Educator.\\
Users interact with the system by interfacing with a Content Delivery Network (CDN) that's a Single Page App (SPA) site which enables the forwarding of appropriate requests to an API Gateway.
The user is completely unaware of the microservices structure used to operate the system.\\
Each microservice realizes a service useful for the fulfillment of the single functionality that the CKB Platform is to provide.
Microservices do not communicate directly with each other but instead use an API Gateway.
All requests go through the API Gateway which realizes complex services by sending requests to the several available microservices.\\
The CKB platform can be abstracted into 5 main areas:
\begin{enumerate}

    \item Battles, Tournaments, Teams and Scores
    \item Notifications
    \item Badges
    \item Compiling
    \item Analysing

\end{enumerate}

\comment{mettere disegnino carino ad alto livello}\\

These sections contain modules with functionality related to the same domain independent of each other, consequently the areas will be implemented by different microservices ensuring low coupling and high cohesion.\\
The choice to use microservice architecture allows flexibility and scalability because all services can be duplicated by not having to interact with databases shared among microservices that could become bottlenecks of the system.
In particular, the API Gateway is responsible for performing load balancing and being stateless, more or less resources can be dynamically allocated depending on the load of requests that are received.\\
In addition, the components involved in sending notifications, compiling sources, running tests, and analyzing them perform more or less complex operations that may take a few moments of execution time.
Consequently, they will be implemented in such a way as to be asynchronous with the use of message queues avoiding making users wait for a long time (more details in later sections).


\section{Component view}
\begin{adjustbox}{
        max size={\textwidth}{\textheightwithcaption{1}},
        caption={Component view diagram},
        label={fig:Component view diagram},
        figure=H}
    \centering
    \puml{puml/component-view}
\end{adjustbox}

\subsection{Server Components}
The server components contain the business logic needed to provide the functionalities of the application to the clients,
by responding to requests made by Client Components as well as perform tasks based on interaction with the external
platforms used by the clients (such as GitHub). There are 5 main components, which corresponds to the main areas identified
above, as well as a few additional components:
\begin{itemize}
    \item \textbf{User Service} {-} implements the authentication as well as keeps track of both students and educators by making
          use of 3rd-party authentication services provided by the institutions.
    \item \textbf{Platform Service} {-} implements all the logic related to creating and managing battles and tournaments, including
          calculating the score and keeping track of the leaderboards
    \item \textbf{Badge Service} {-} implements all the logic related to the evaluation and assignment of badges, including the gathering
          of information needed in order to do that (for example Git or GH metadata such as number of commits by students, etc)
          as well as the actual evaluation of JavaScript code which gets used by educators in order to express the logic of
          specific badges
    \item \textbf{Build and Test Service} {-} implements the Continuous Integration aspect, by building students projects and
          running them againsts tests for each push
    \item \textbf{Static Analysis Service} {-} abstracts away the interaction with SonarCloud, the 3rd party service which will
          perform the actual static analysis. While SonarCloud already exposes their own API for interaction, it is cumbersome
          (as it requires registering a webhook \cite{SonarCloudWh}, importing the project from the GH slug \cite{SonarCloudGh},
          specifying the required analysis metrics in a quality profile \cite{SonarCloudQp} and waiting for a response on the hook).
          Therefore, by abstracting it away we can use it much more easily and, additionally, make it much less inconvenient to
          replace it should the need arise.
    \item \textbf{Notification Service} {-} responsible for contacting external notification APIs
          (such as Google's Firebase Cloud Messaging or Apple Push Notification Service) in order to deliver push notifications
          to client devices
    \item \textbf{Website CDN} {-} responsible for serving the static files of the SPA to the clients
    \item \textbf{API Gateway} {-} responsible for connecting together all microservices, it is the component that provides a REST
          interface externally, which the clients will send requests to, which will be implemented by a series of internal calls
          using gRPC \cite{gRPC} to all the other microservices in order to make responses. The component does not implement a specific
          functionality per se, but is in charge of providing an interface to the clients as well as enforcing the correct usage
          of said interface.
\end{itemize}

\subsection{Data Components}
Server Components make use of 3 distinct DBMSes, each with its own schema.
\begin{itemize}
    \item \textbf{User DB} {-} saves data related to the users of the platform. Its ER schema is not shown
          here as it would depend on the 3rd-party service, but it would at leats require for each student and
          educator a unique identifier, as well as human readable identifier to allow users to lookup each other
          (so name and surname or an email). Educators and Students are saved in different tables.
          \pagebreak
    \item \textbf{Platform DB}
          \begin{adjustbox}{
                  max size={\textwidth}{\textheightwithcaption{1}},
                  caption={Platform ER},
                  label={fig:Platform ER},
                  figure=H}
              \centering
              \puml{puml/platform-db}
          \end{adjustbox}
          \pagebreak
    \item \textbf{Badge DB}
          \begin{adjustbox}{
                  max size={\textwidth}{\textheightwithcaption{1}},
                  caption={Badges ER},
                  label={fig:Badges ER},
                  figure=H}
              \centering
              \puml{puml/badges-db}
          \end{adjustbox}
\end{itemize}
\pagebreak

\section{Deployment view}
\begin{adjustbox}{
        max size={\textwidth}{\textheightwithcaption{1}},
        caption={Deployment view diagram},
        label={fig:Deployment view diagram},
        figure=H}
    \centering
    \puml{puml/deployment-diagram}
\end{adjustbox}
\pagebreak

\section{Runtime view}
The following sequence diagrams represent the dynamics of interaction between components.\\
Sequence diagrams from \ref{{RW1}} to \ref{{RW9}} are the realizations of the corresponding use cases in the RASD document.\\
Sequence diagrams from \ref{{RW0.1}} to \ref{{RW0.3}} are common part that have been extracted for simplicity and are not shown in other diagrams.\\
In the first three sequence diagrams, the user could be a student or an educator.

\begin{enumerate}[label=\textbf{RW\arabic*}:,ref=RW\arabic*,leftmargin=1.3cm]
    \item[]
          \begin{enumerate}[label=\textbf{RW\arabic{enumi}.\arabic*}:,ref=RW\arabic{enumi}.\arabic*,leftmargin=0.5cm]
              \labelledsubitem{
                  \textbf{}
                  \begin{adjustbox}{
                          max size={\textwidth}{\textheightwithcaption{1}},
                          caption={User gets list of tournaments},
                          label={fig:User gets list of tournaments},
                          figure=H}
                      \centering
                      \puml{puml/rw0.1}
                  \end{adjustbox}
                  \pagebreak
              }
              \labelledsubitem{
                  \textbf{}
                  \begin{adjustbox}{
                          max size={\textwidth}{\textheightwithcaption{1}},
                          caption={User gets tournament details},
                          label={fig:User gets tournament details},
                          figure=H}
                      \centering
                      \puml{puml/rw0.2}
                  \end{adjustbox}
              }
              \labelledsubitem{
                  \textbf{}
                  \begin{adjustbox}{
                          max size={\textwidth}{\textheightwithcaption{1}},
                          caption={User gets battle details},
                          label={fig:User gets battle details},
                          figure=H}
                      \centering
                      \puml{puml/rw0.3}
                  \end{adjustbox}
                  \pagebreak
              }
          \end{enumerate}

          \labelleditem{
              \textbf{}
              \begin{adjustbox}{
                      max size={\textwidth}{\textheightwithcaption{1}},
                      caption={Educator creates a new Tournament},
                      label={fig:Educator creates a new Tournament},
                      figure=H}
                  \centering
                  \puml{puml/rw1}
              \end{adjustbox}
              \pagebreak
          }
          \labelleditem{
              \textbf{}
              \begin{adjustbox}{
                      max size={\textwidth}{\textheightwithcaption{1}},
                      caption={Educator creates a new badge},
                      label={fig:Educator creates a new badge},
                      figure=H}
                  \centering
                  \puml{puml/rw2}
              \end{adjustbox}
              \pagebreak
          }
          \labelleditem{
              \textbf{}
              \begin{adjustbox}{
                      max size={\textwidth}{\textheightwithcaption{1}},
                      caption={Educator creates a new Battle for an Existing Tournament},
                      label={fig:Educator creates a new Battle for an Existing Tournament},
                      figure=H}
                  \centering
                  \puml{puml/rw3}
              \end{adjustbox}
              \pagebreak
          }
          \labelleditem{
              \textbf{}
              \begin{adjustbox}{
                      max size={\textwidth}{\textheightwithcaption{1}},
                      caption={Student joins to an existing Tournament by receiving a notification},
                      label={fig:Student joins to an existing Tournament by receiving a notification},
                      figure=H}
                  \centering
                  \puml{puml/rw4}
              \end{adjustbox}
              \pagebreak
          }
          \labelleditem{
              \textbf{}
              \begin{adjustbox}{
                      max size={\textwidth}{\textheightwithcaption{1}},
                      figure=H}
                  \centering
                  \puml{puml/rw5-part1}
              \end{adjustbox}
              \pagebreak
              \begin{adjustbox}{
                      max size={\textwidth}{\textheightwithcaption{1}},
                      caption={Students create a team for a tournament battle},
                      label={fig:Students create a team for a tournament battle},
                      figure=H}
                  \centering
                  \puml{puml/rw5-part2}
              \end{adjustbox}
              \pagebreak
          }
          \labelleditem{
              \textbf{}
              \begin{adjustbox}{
                      max size={\textwidth}{\textheightwithcaption{1}},
                      caption={Student forks the repository},
                      label={fig:Student forks the repository},
                      figure=H}
                  \centering
                  \puml{puml/rw6}
              \end{adjustbox}
              \pagebreak
          }
          \labelleditem{
              \textbf{}
              \begin{adjustbox}{
                      max size={\textwidth}{\textheightwithcaption{2}},
                      caption={Student pushes and triggers automatic evaluation},
                      label={fig:Student pushes and triggers automatic evaluation},
                      figure=H}
                  \centering
                  \puml{puml/rw7}
              \end{adjustbox}
              \pagebreak
          }
          \labelleditem{
              \textbf{}
              \begin{adjustbox}{
                      max size={\textwidth}{\textheightwithcaption{1}},
                      caption={Educator manually evaluates teams},
                      label={fig:Educator manually evaluates teams},
                      figure=H}
                  \centering
                  \puml{puml/rw8}
              \end{adjustbox}
              \pagebreak
          }
          \labelleditem{
              \textbf{}
              \begin{adjustbox}{
                      max size={\textwidth}{\textheightwithcaption{1}},
                      caption={Educator closes a tournament},
                      label={fig:Educator closes a tournament},
                      figure=H}
                  \centering
                  \puml{puml/rw9}
              \end{adjustbox}
              \pagebreak
          }
\end{enumerate}

\section{Component interfaces}
Here the most relevant interfaces exposed by components are described, including all the operations
seen in the previous diagrams:
\begin{itemize}
    \item \textbf{API Gateway}
          \begin{itemize}
              \item \textbf{getListOfTournaments(authToken: String): List<SimpleTournament>}
              \item \textbf{getBattleDetails(authToken: String, battleId: ID): List<Battle>}
              \item \textbf{searchEducator(authToken: String, name: String?, surname: String?, \ldots): ID?}
              \item \textbf{searchStudent(authToken: String, name: String?, surname: String?, \ldots): ID?}
              \item \textbf{createTournament(authToken: String, tournamentInfo: NewTournament): ID}
              \item \textbf{createBadge(authToken: String, badgeInfo: NewBadge): ID}
              \item \textbf{createNewBattle(authToken: String, tournementId: ID, battleInfo: Battle): ID}
              \item \textbf{enrollInTournament(authToken: String, tournementId: ID): void}
              \item \textbf{createTeam(authToken: String, battleId: ID): ID}
              \item \textbf{invite(authToken: String, newTeamId: ID, invitedStudentId: ID): void}
              \item \textbf{acceptInvitation(authToken: String, invitation: ID): void}
              \item \textbf{rejectInvitation(authToken: String, invitation: ID): void}
              \item \textbf{insertRepoSlug(authToken: String, battleId: ID, repo: Slug): void }
              \item \textbf{newPush(repo: Slug): void }
              \item \textbf{assignGrade(authToken: String, battleId: ID, teamId: ID, grade: Int): void }
              \item \textbf{closeTournament(authToken: String, tournamentId: ID): void }
              \item \textbf{closeBattle(authToken: String, battleId: ID): void }
          \end{itemize}
    \item \textbf{User Interface}
          \begin{itemize}
              \item \textbf{validateEducatorToken(authToken: String): ID?}
              \item \textbf{validateStudentToken(authToken: String): ID?}
              \item \textbf{searchEducator(name: String?, surname: String?, \ldots): ID?}
              \item \textbf{searchStudent(name: String?, surname: String?, \ldots): ID?}
              \item \textbf{checkEducatorIds(educatorsIds: ID[]): Bool}
              \item \textbf{checkStudentId(studentId: ID): Bool}
          \end{itemize}
    \item \textbf{Platform Interface}
          \begin{itemize}
              \item \textbf{getListOfTournaments(educatorId: ID): List<SimpleTournament>}
              \item \textbf{getTournamentDetails(tournamentId: ID): List<Tournament>}
              \item \textbf{getBattleDetails(battleId: ID): List<Battle>}
              \item \textbf{createTournament(creatorId: ID, tournamentInfo: NewTournament): ID}
              \item \textbf{createNewBattle(tournementId: ID, battleInfo: NewBattle): ID}
              \item \textbf{enrollInTournament(studentId: ID, tournementId: ID): void}
              \item \textbf{createTeam(battleId: ID, studentId: ID): ID}
              \item \textbf{invite(newTeamId: ID, invitedStudentId: ID): void}
              \item \textbf{acceptInvitation(invitation: ID): void}
              \item \textbf{rejectInvitation(invitation: ID): void}
              \item \textbf{insertRepoSlug(studentId: ID, battleId: ID, repo: Slug): void }
              \item \textbf{findBattleAndTeamFor(repo: Slug): {battleId: ID, teamId: ID}? }
              \item \textbf{calculateNewScore(battleId: ID, teamId: ID, testsReport: TestReport, stReport: StaReport): void }
              \item \textbf{assignGrade(battleId: ID, teamId: ID, grade: Int): void }
              \item \textbf{closeTournament(tournamentId: ID): void }
              \item \textbf{closeBattle(battleId: ID): void }
          \end{itemize}
    \item \textbf{Badge Interface}
          \begin{itemize}
              \item \textbf{createBadge(creatorId: ID, badgeInfo: NewBadge): ID}
              \item \textbf{assignBadges(tournament: Tournament): void}
          \end{itemize}
    \item \textbf{Notification Interface}
          \begin{itemize}
              \item
          \end{itemize}
    \item \textbf{Static Analysis Interface}
          \begin{itemize}
              \item \textbf{buildAndTest(repo: Slug): TestReport}
          \end{itemize}
    \item \textbf{Build and Test Interface}
          \begin{itemize}
              \item \textbf{analyse(repo: Slug): StaReport}
          \end{itemize}
\end{itemize}

Types:
\begin{itemize}
    \item \textbf{ID} {-} a unique identifier for a given object, could be a self-incrementing integer or a UUID (depends on the choosen DBMS)
    \item \textbf{Slug} {-} a String identifying a specific GitHub repository, in the format '<username>/<repository>' as can be seen
          in a GitHub project URL
    \item \textbf{SimpleTournament}
          \begin{lstlisting}
{
    id: ID,
    name: String,
    status: Subscribing | InProgress | Closed,
} 
          \end{lstlisting}
    \item \textbf{Tournament}
          \begin{lstlisting}
{
    id: ID,
    name: String,
    status: Subscribing | InProgress | Closed,
    subscriptionDeadline: DateTime,
    enrolledStudents: { id: ID, name: String, score: Int }[],
    battles: { id: ID }[],
} 
          \end{lstlisting}
    \item \textbf{NewTournament}
          \begin{lstlisting}
{
    name: String,
    subscriptionDeadline: DateTime,
    authorizedEducators: { id: ID },
    badges: { id: ID }[],
} 
          \end{lstlisting}
    \item \textbf{Battle}
          \begin{lstlisting}
{
    name: String,
    description: String,
    status: Registration | Submission | Consolidation | Done,
    minStudents: Int,
    maxStudents: Int,
    registrationDeadline: DateTime,
    submissionDeadline: DateTime,
    requiresManualEvaluation: Bool,
    repo: Slug?,
    teams: { id: ID, students: { id: ID }[], score: Int }[],
} 
          \end{lstlisting}
    \item \textbf{NewBattle}
          \begin{lstlisting}
{
    title: String,
    condition: String,
} 
          \end{lstlisting}
    \item \textbf{NewBadge}
          \begin{lstlisting}
{
    title: String,
    condition: String,
} 
          \end{lstlisting}
    \item \textbf{TestReport}
          \begin{lstlisting}
{
    passed: Int,
    failed: Int,
} 
          \end{lstlisting}
    \item \textbf{StaReport}
          \begin{lstlisting}
Map<{ metricName: String }, { score: Int }> 
          \end{lstlisting}
\end{itemize}

\section{Selected architectural styles and patterns}

\section{Other design decisions}
