\section{External Interface Requirements}
\section{Functional Requirements}
\subsection{Use case diagrams}
\subsection{Use cases}
%Oppure facciamo tre sezioni separate per tabella, seq diagram e activity diagram,
% ma così mi sembra più leggibile.
\begin{enumerate}[label=\textbf{UC\arabic*}:,leftmargin=1.3cm]
    \item 
    \item 
    \item 
    \item 
    \item 
    \item 
    \item 
    \item \textbf{Closing a tournament}
          \begin{table}[H]
              %\caption*{\textbf{Title}}
              \centering
              \begin{tabular}{|l|p{11.9cm}|}
                  \hline
                  \textbf{Name}            & Closing a tournament                                 \\\hline
                  \textbf{Actor}           & Educator, Students                                   \\\hline
                  \textbf{Entry condition} &
                  \begin{itemize}
                      \item Educator has logged in
                      \item Educator has permission to modify the tournament
                      \item The tournament is in the consolidation stage
                  \end{itemize}                           \\\hline
                  \textbf{Event flow}      &
                  \begin{enumerate}
                      \item Educator click the \emph{close} button in the tournament page.
                      \item The platform calculate the final tournament rank.
                      \item The platform notifies students who participated in the tournament.
                      \item The platform assign badges to the student who fulfilled the rules.
                  \end{enumerate}         \\\hline
                  \textbf{Exit condition}  & The tournament is closed and the bedges are assigned \\\hline
                  %\textbf{Exception}       &                                                      \\\hline
                  %\textbf{Special reqs}    &                                                      \\\hline
              \end{tabular}
              \caption{Closing a tournament use case.}
              \label{table:Closing a tournament use case}
          \end{table}

          \puml{puml/UC8}

\end{enumerate}
\subsection{Requirements mapping}
\section{Performance Requirements}

\section{Design Constraints}
\subsection{Standards Compliance}
The CodeKataBattle (CKB) platform is designed with the goal of strictly adhering to several standards to ensure the quality, security and interoperability of the system.\\ The standards to which CKB adheres include:

\begin{enumerate}
\item HTTPS Protocol Compliance\\
Implementation of the HTTPS protocol according to the cryptographic standards established by IETF (Internet Engineering Task Force).

\item Accessibility Standards\\
Compliance with Web Content Accessibility Guidelines (WCAG) to ensure web accessibility.

\item Security Standards\\
Implementation of security best practices as defined by OWASP (Open Web Application Security Project) and NIST (National Institute of Standards and Technology).
Such as password storage encryption with HASH512 + Salt, SSL Certificate and end-to-end communication encryption.


\item API Interoperability\\
Use of open standards for API design, such as RESTful API and adherence to specifications such as OpenAPI (Swagger).


\item Coding Standards\\
The platform follows universally accepted coding guidelines for the main programming language used in system development (e.g., Java or Python). 
Compliance with the coding guidelines for the main programming language (PEP 8 for Python, Java Coding Conventions for Java).


\item Compliance with Privacy Regulations\\
Compliance with privacy laws such as the General Data Protection Regulation (GDPR) for European citizens.
\end{enumerate}

\subsection{Hardware limitations}
Each student and educator must have an electronic device such as a computer or smartphone with an internet connection to 
access the CKB Platform.
Students must be able to clone GitHub repositories on their device to participate in the various battles and accordingly must have a physical storage with sufficient space to store the sources and data of the various projects.
The device used must allow the student to view, edit, process and execute the sources.
The devices used by students and educators enable them to receive notifications.

\section{Software System Attributes}
\subsection{Reliability}
The CKB platform does not manage critical operations.
If an operation fails, it can be re-executed without any particular consequences.
For example, if the evaluation of an exercise fails, students can resubmit their code.
It is therefore reasonable to have a failure rate around 1\%.

\subsection{Availability}
The platform should have the lowest downtime possible during the day.
A few hours of downtime are allowed between 1am and 5am, when student are usually not studying.
The platform shall therefore guarantee 99\% (two-nines) of availability, so that 3.65 days of downtime per year are allowed.

\subsection{Security}
Communication between the user and the CBK platform is encrypted with HTTPS. %Forse non posso dirlo qua(?)
Furthermore, users must only be able to perform operations that they are authorized to do.
For example, a student must not be able to change his grade.

\subsection{Maintainability}
The system should be divided in scalable and reusable modules,
which will be easier to maintain and replace in case of failure.
Ordinary maintenance, for bug fixes and improvements, will be scheduled during night time, when the user traffic is minimal.

\subsection{Portability}
The CKD platform does not require any particular hardware or software.
It must be accessible from any operating system with a modern web browser.
A mobile application that allows users to see the state of the battles can also be developed.
Since the mobile app does not require any special function, a non-native approach can be adopted.
It is therefore possible to use cross-platform development tools, which can speed up the development process.