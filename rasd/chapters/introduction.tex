% !TeX root = ../rasd.tex
\section{Purpose}
The purpose of the system is to help students improve their software development skills by joining Code Kata Battles.
A Code Kata Battle is a programming exercise created by an educator that contains a set of test cases.
Students have to provide a solution that passes the tests, using a test-driven approach.
Students can try as many times as they want, each time a solution is submitted, the tests are executed and the score is calculated.

The goals of the CKB platform are:

\begin{enumerate}[label=\textbf{G\arabic*}:,ref=G\arabic*,leftmargin=1.3cm]
    \labelleditem{Educators can organize tournaments of programming exercises aimed at improving their students' skills.}
    \labelleditem{Educators can organize battles in a tournament they have created.}
    \labelleditem{Educators can aid their colleagues in creating battles.}
    \labelleditem{Educator can decide how to score a battle they have created.}
    \labelleditem{Students can be notified about new tournaments.}
    \labelleditem{Students can subscribe to tournaments.}
    \labelleditem{Students can be notified about new battles in tournaments they are subscribed to.}
    \labelleditem{Students can form teams to take part in a battle.}
    \labelleditem{Teams of students can join a battle.}
    \labelleditem{Students and educators can monitor the progress of all students in battles and tournaments.}
    \labelleditem{Educators can manage students’ gamification badges.}
    \labelleditem{Students can receive badges depending on the rules they have fulfilled.}
    \labelleditem{Students and educators can see badges collected by a student.}
\end{enumerate}

\pagebreak

\section{Scope}
The platform is intended to be used in a school or university context.
Educators are the professors who teach in the school.\\
The platform interacts with an existing login system (SSO), responsible for authenticating users.

The platform will allow an educator to create tournaments and battles within the context of a tournament he has created.
The educator can also grant other colleagues the permission to create battles within the context of his tournaments.

The creator of the tournament can also create badges.
Each badge has a set of rules that must be fulfilled to achieve it.
The rules are simple expressions on variables that returns a boolean value.
For example:
\begin{lstlisting}
    tot_attended_battles > 0
\end{lstlisting}
or
\begin{lstlisting}
    tot_commits_student == max_tot_commits
\end{lstlisting}
where \lstinline{tot_attended_battles}, \lstinline{tot_commits_student} and \lstinline{max_tot_commits} are pre-defined variables.\\
To create more complex rule and avoid duplicated code, educators can create new variables.
In particular, they can use an interpreted language to define new variables based on the existing ones.
For example, if the platform provides the variables \lstinline{tot_attended_battles}and \lstinline{tot_battles},
educators can create a new variable \lstinline{tot_missed_battles} as
\begin{lstlisting}[language=java, morekeywords={var}, keywordstyle=\color{blue}]
    var tot_missed_battles = tot_battles - tot_attended_battles;
\end{lstlisting}

To create a battle, the educator needs to upload test cases and build automation scripts.
Uploading test cases he can decide if each test is public or private.
Public tests are provided to students, so that they can run them locally.
Private tests are used for automatic evaluation, but they are not provided to students.
In this way educator are sure that students are not writing code that only pass specific tests, but actually solves the exercise.

When the registration deadline of a battle expires, the platform creates a GitHub repository containing build automation scripts and public tests provided by the educator.
All subscribed students receive a notification.
One student per group forks the repository and adds his teammates as collaborators.\\
The forked repo contains also a workflow that triggers the CKB platform on each push.
The student needs to enable the GitHub Actions to make it work.
The student also has to insert the link of his repository in the CKB platform, so that only authorized repositories can trigger the platform.

When the submission deadline of a battle expires, if the educator enabled manual evaluation, there is a consolidation stage.
During this stage, the educator who create the battle uses the platform to go through the sources produced by each team to assign his/her score.
After grading all groups, he can close the consolidation stage.\\
Otherwise, if manual evaluation is not required, when the submission deadline expires, the battle is automatically closed.

After the battle is closed, the final battle rank becomes available and all students who participated are notified.

After all battles in a tournament are closed, the educator who create the tournament can close it.

The following table describes world, shared and machine phenomena.
\begin{center} %Limits the scope of \rowcolors
    \rowcolors{2}{gray!25}{white}
    \begin{longtable}{|p{8.7cm}|p{3cm}|p{3cm}|}
        \caption[Phenomena Table]{}
        \label{table:phenomena}
        \endlastfoot
        \hline
        \rowcolor{gray!50}
        \textbf{Phenomena}                                                                                                                    & \textbf{Controlled by} & \textbf{Shared} \\ \hline
        A user wants to log in to the platform                                                                                                & W                      & N               \\ \hline
        An educator wants to create a tournament                                                                                              & W                      & N               \\ \hline
        A student wants to take part in a tournament                                                                                          & W                      & N               \\ \hline
        A student wants to invite other students in a tournament                                                                              & W                      & N               \\ \hline
        A student wants to see the ranking in a battle                                                                                        & W                      & N               \\ \hline
        An educator wants to see the ranking in a battle                                                                                      & W                      & N               \\ \hline
        A user logs in                                                                                                                        & W                      & Y               \\ \hline
        An educator creates a tournament                                                                                                      & W                      & Y               \\ \hline
        The system notifies students about a new tournament                                                                                   & M                      & Y               \\ \hline
        An educator grants permissions for a tournament to another educator                                                                   & W                      & Y               \\ \hline
        A student subscribes to a tournament                                                                                                  & W                      & Y               \\ \hline
        An educator creates a battle as part of a tournament he has permissions for                                                           & W                      & Y               \\ \hline
        The system notifies students subscribed to a tournament about a new battle in said tournament                                         & M                      & Y               \\ \hline
        A student invites another student to take part in a battle with him                                                                   & W                      & Y               \\ \hline
        A student accepts the invitation of another student to take part in a battle with him                                                 & W                      & Y               \\ \hline
        A student joins a battle on their own                                                                                                 & W                      & Y               \\ \hline
        The registration deadline of a battle expires                                                                                         & W                      & N               \\ \hline
        The system sends the link to a GitHub repository containing the code for a battle to all students who are members of subscribed teams & M                      & Y               \\ \hline
        A student forks the GitHub repository                                                                                                 & W                      & N               \\ \hline
        A student sets up the automated workflow                                                                                              & W                      & N               \\ \hline
        A student writes code                                                                                                                 & W                      & N               \\ \hline
        A student pushes a new commit in their repository main branch                                                                         & W                      & Y               \\ \hline
        The system pulls the latest sources, analyses them, and runs the tests                                                                & M                      & N               \\ \hline
        The system calculates the battle score of the team.                                                                                   & M                      & N               \\ \hline
        A student or an educator visualize the current rank of a battle                                                                       & W                      & Y               \\ \hline
        The submission deadline expires and the consolidation stage starts                                                                    & W                      & N               \\ \hline
        An educator manually evaluates the submission of a team for a battle he has created                                                   & W                      & Y               \\ \hline
        An educator closes the consolidation stage                                                                                            & W                      & Y               \\ \hline
        The system notifies students taking part in a battle that the final battle rank is available                                          & M                      & Y               \\ \hline
        An educator closes a tournament                                                                                                       & W                      & Y               \\ \hline
        The platform updates the personal tournament score of each student                                                                    & M                      & N               \\ \hline
        The system notifies students taking part in a tournament that the final tournament rank is available                                  & M                      & Y               \\ \hline
        Any subscribed user visualizes the list of current tournaments                                                                        & W                      & Y               \\ \hline
        Any subscribed user visualizes the ranking of a tournament                                                                            & W                      & Y               \\ \hline
        An educator defines a badge                                                                                                           & W                      & Y               \\ \hline
        A student achieves a badge                                                                                                            & W                      & Y               \\ \hline
        Any subscribed user visualizes the badges achieved by a student                                                                       & W                      & Y               \\ \hline
    \end{longtable}
\end{center}

\pagebreak

\section{Definitions, Acronyms, Abbreviations}

\subsection{Definitions}
\begin{description}[leftmargin=0pt]
    \item[CodeKataBattle (CKB):] A platform designed to help students enhance their software development skills through collaborative coding exercises known as code kata battles.

    \item[Kata:] In karate, a kata is an exercise that involves repetitive practice to improve coding skills. Code Kata brings this concept to software development as a form of deliberate practice.

    \item[Test-First Approach:] A development methodology in which developers write tests before implementing the corresponding code, ensuring that the code meets specified requirements and functions as expected.

    \item[GitHub:] A web-based platform that provides version control and collaboration features using Git, facilitating software development projects. \cite{GitHub}

    \item[API (Application Programming Interface):] A set of rules and protocols that allows different software applications to communicate and interact with each other.

    \item[Static Analysis Tools:] Software tools that analyze source code without execution, providing insights into code quality, security, reliability, and maintainability.

    \item[Gamification Badges:] In the context of the CKB platform, these are rewards representing individual achievements within tournaments. Educators define badges with titles and rules.

    \item[Tournament:] A series of Code Kata Battles organized by educators on the CKB platform, providing students with opportunities to compete and improve their coding skills.

    \item[Battle:] A coding exercise within a tournament that includes a textual description, a software project with test cases, and build automation scripts. Teams aim to complete the project by following a test-first approach.

    \item[Deadline:] A specified date or time by which certain actions, such as registration or submission, must be completed.

    \item[Score:] A numerical representation of a team's performance in a battle, determined by various factors such as test case success, timeliness, and code quality.

    \item[Rank:] The position of a team or student relative to others in a competition, often determined by their scores or performance.

    \item[Notification:] A message or alert that informs users about events, changes, or updates within the platform.

    \item[Collaborators:] Educators who are granted permission to create battles within a specific tournament.

    \item[Build Automation Scripts:] Scripts that automate the process of building, compiling, and testing code.

    \item[Git:] A distributed version control system used for tracking changes in source code during software development. \cite{Git}

    \item[Git Repository:] A storage location for a Git project, containing all files and the version history of those files.

    \item[GitHub Actions:] Automated workflows defined on GitHub that can perform various tasks such as building, testing, and deploying projects.

    \item[Personal Tournament Score:] The cumulative score of a student in all battles within a specific tournament, reflecting their overall performance.

    \item[Gamification:] The application of game elements, such as badges and rewards, in non-game contexts to motivate and engage users.

    \item[API Calls:] Requests made by the platform to external services, such as GitHub, using predefined interfaces to retrieve or send data.

    \item[Dynamic Ranking:] A continuously updated ranking that changes based on ongoing events or activities, providing real-time insights into participants' standings.

\end{description}

\subsection{Acronyms}
\begin{description}[leftmargin=0pt]
    \item[CKB:] CodeKataBattle
    \item[API:] Application Programming Interface
    \item[SAT:] Static Analysis Tools
    \item[GAB:] Gamification Badges
    \item[BAS:] Build Automation Scripts
    \item[GR:] Git Repository
    \item[GH:] GitHub
    \item[PTS:] Personal Tournament Score
    \item[DR:] Dynamic Ranking
    \item[DL:] Deadline
    \item[TDD:] Test-Driven Development
    \item[SSO:] Single Sign On
    \item[IDE:] Integrated Development Environment
\end{description}

\subsection{Abbreviations}
\begin{description}[leftmargin=0pt]
    \item[e.g.:] For example
    \item[e.g.:] That is
    \item[repo:] Repository
    \item[Gn:] Goal number n
    \item[Dn:] Domain assumption number n
    \item[Rn:] Requirement number n
    \item[UCn:] Use case number n
\end{description}

\section{Revision history}

\begin{itemize}
    \item \textbf{1.0} (21st December 2023) {-} Initial release
    \item \textbf{1.1} (6th January 2024)
          \begin{itemize}
              \item Fixed a few cardinalities in model class diagram
              \item Minor fix to a method name in UC5 diagram
              \item Fixed all diagrams centering
          \end{itemize}
    \item \textbf{1.2} (26th January 2024)
          \begin{itemize}
              \item Fixed relationships in model class diagram
              \item Parallelize some activities in UC7 diagram
          \end{itemize}
\end{itemize}

\section{Reference Documents}

\begin{description}[leftmargin=0pt]
    \item[Specification document:] \emph{"Assignment RDD AY 2023-2024"}
    \item[ISO/IEC/IEEE 29148 (Nov 2018)]- International Standard - Systems and software engineering -- Life cycle processes -- Requirements engineering:\\\url{https://doi.org/10.1109/IEEESTD.2018.8559686}
    \item[The world and the machine]- Michael Jackson, 1995:\\\url{https://doi.org/10.1145/225014.225041}
    \item[UML official specification:] \url{https://www.omg.org/spec/UML/}
    \item[Alloy official documentation:] \url{https://alloytools.org/documentation.html}
\end{description}


\section{Document Structure}

\begin{enumerate}
    \item \textbf{Section 1: Introduction} \\
          This section gives a brief description of the problem, as well as the purpose and the scope of
          the system.
          It also contains the list of definitions, acronyms and abbreviations that might be encountered
          while reading the document.
          Additionally, there's the revision history of the document, which keeps track of the various
          version, their release date and their changes.
    \item \textbf{Section 2: Overall Description} \\
          This section contains a description of the entirety of the problem, detailing its domain,
          including assumptions that must hold true, as well as how its state changes in time
          and possible usage scenarios.
    \item \textbf{Section 3: Specific Requirements} \\
          This section describes the specific requirements of the system. It provides a detailed analysis
          of the external interfaces, both user-facing as well as hardware and software, and presents the
          performance requirements, standards compliance and software system attributes.
          Additionally, it describes in depth the functional requirements of the system and details the
          use cases by means of diagrams, mapping goals, requirements and domains and tracing each use
          case to the requirements it satisfies.
    \item \textbf{Section 4: Formal Analysis using Alloy} \\
          This section provides a formal analysis using the alloy language, in order
          to prove the correctness and soundness of the system described in the previous sections.
\end{enumerate}