% A LaTeX template for MSc Thesis submissions to 
% Politecnico di Milano (PoliMi) - School of Industrial and Information Engineering
%
% S. Bonetti, A. Gruttadauria, G. Mescolini, A. Zingaro
% e-mail: template-tesi-ingind@polimi.it
%
% Last Revision: October 2021
%
% Copyright 2021 Politecnico di Milano, Italy. NC-BY

\documentclass{config/PoliMi3i_thesis}

%------------------------------------------------------------------------------
%	REQUIRED PACKAGES AND  CONFIGURATIONS
%------------------------------------------------------------------------------

% Uncomment to show margins
%\usepackage{showframe}

% PACKAGE FOR CUSTOM LISTS
\usepackage{enumitem}

% CONFIGURATIONS
\usepackage{parskip} % For paragraph layout
\usepackage{setspace} % For using single or double spacing
\usepackage{emptypage} % To insert empty pages
\usepackage{multicol} % To write in multiple columns (executive summary)
\setlength\columnsep{15pt} % Column separation in executive summary
\setlength\parindent{0pt} % Indentation
\raggedbottom

% PACKAGES FOR TITLES
\usepackage{titlesec}
% \titlespacing{\section}{left spacing}{before spacing}{after spacing}
\titlespacing{\chapter}{0pt}{0ex}{8ex}
\titlespacing{\section}{0pt}{3.3ex}{2ex}
\titlespacing{\subsection}{0pt}{3.3ex}{1.65ex}
\titlespacing{\subsubsection}{0pt}{3.3ex}{1ex}
\usepackage{color}

% PACKAGES FOR LANGUAGE AND FONT
\usepackage[english]{babel} % The document is in English  
\usepackage[utf8]{inputenc} % UTF8 encoding
\usepackage[T1]{fontenc} % Font encoding
\usepackage[11pt]{moresize} % Big fonts

% PACKAGES FOR IMAGES
\usepackage{graphicx}
\usepackage{transparent} % Enables transparent images
\usepackage{eso-pic} % For the background picture on the title page
\usepackage{subfig} % Numbered and caption subfigures using \subfloat.
\usepackage{tikz} % A package for high-quality hand-made figures.
\usetikzlibrary{}
\graphicspath{{./images/}} % Directory of the images
\usepackage{caption} % Coloured captions
\usepackage{xcolor} % Coloured captions
\usepackage{amsthm,thmtools,xcolor} % Coloured "Theorem"
\usepackage{float}
\ifdefined\emitpumldiagrams{}
    \usepackage{svg} % svgs
    \pdfsuppresswarningpagegroup=1 % disable svg export warnings
\fi
\usepackage{adjustbox} % adjust svg size to fit
 
% STANDARD MATH PACKAGES
\usepackage{amsmath}
\usepackage{amsthm}
\usepackage{amssymb}
\usepackage{amsfonts}
\usepackage{bm}
\usepackage[overload]{empheq} % For braced-style systems of equations.
\usepackage{fix-cm} % To override original LaTeX restrictions on sizes

% PACKAGES FOR TABLES
\usepackage{tabularx}
\usepackage{longtable} % Tables that can span several pages
\usepackage{colortbl}
\usepackage{multirow}

% PACKAGES FOR ALGORITHMS (PSEUDO-CODE)
\usepackage{algorithm}
\usepackage{algorithmic}
\usepackage{listings}
\usepackage{config/alloy-style}

% PACKAGES FOR REFERENCES & BIBLIOGRAPHY
\usepackage[colorlinks=true,linkcolor=black,anchorcolor=black,citecolor=black,filecolor=black,menucolor=black,runcolor=black,urlcolor=black]{hyperref} % Adds clickable links at references
\usepackage{cleveref}
\usepackage[square, numbers, sort&compress]{natbib} % Square brackets, citing references with numbers, citations sorted by appearance in the text and compressed
\bibliographystyle{abbrvnat} % You may use a different style adapted to your field

% OTHER PACKAGES
\usepackage{pdfpages} % To include a pdf file
\usepackage{afterpage}
\usepackage{lipsum} % DUMMY PACKAGE
\usepackage{fancyhdr} % For the headers
\usepackage{tabu}
\usepackage[]{underscore} % To be able to use _ in text
\fancyhf{}

% Input of configuration file. Do not change config.tex file unless you really know what you are doing. 
\input{config/config}

%----------------------------------------------------------------------------
%	NEW COMMANDS DEFINED
%----------------------------------------------------------------------------

\newcommand*{\puml}[2][]{%
    \ifdefined\emitpumldiagrams{}
        \immediate\write18{./puml/compile-svg.sh #2}
        \begin{adjustbox}{max size={\textwidth}{\textheight}}
            % inkscapelatex false cause otherwise for some reason inkscape puts white boxes all around text
            \includesvg[inkscapelatex=false, #1]{#2}%
        \end{adjustbox}
    \fi
}

% EXAMPLES OF NEW COMMANDS
\newcommand{\bea}{\begin{eqnarray}} % Shortcut for equation arrays
\newcommand{\eea}{\end{eqnarray}}
\newcommand{\e}[1]{\times 10^{#1}}  % Powers of 10 notation

%----------------------------------------------------------------------------
%	ADD YOUR PACKAGES (be careful of package interaction)
%----------------------------------------------------------------------------

%----------------------------------------------------------------------------
%	ADD YOUR DEFINITIONS AND COMMANDS (be careful of existing commands)
%----------------------------------------------------------------------------

% Some utilities\ldots
\newcommand{\comment}[1]{{\color{red}\(\blacktriangleright\) Comment: #1 \(\blacktriangleleft\)}}
\input{config/highlight}
\input{config/abbrev}

%----------------------------------------------------------------------------
%	BEGIN OF YOUR DOCUMENT
%----------------------------------------------------------------------------

\begin{document}

\fancypagestyle{plain}{%
    \fancyhf{} % Clear all header and footer fields
    \fancyhead[RO,RE]{\thepage} %RO=right odd, RE=right even
    \renewcommand{\headrulewidth}{0pt}
    \renewcommand{\footrulewidth}{0pt}}

%----------------------------------------------------------------------------
%	TITLE PAGE
%----------------------------------------------------------------------------

\pagestyle{empty} % No page numbers
\frontmatter % Use roman page numbering style (i, ii, iii, iv...) for the preamble pages

\puttitle{
    title=Requirement Analysis and Specification, % Title of the thesis
    nameA=Giacomo Orsenigo, % Author Name and Surname
    nameB=Federico Saccani, % Author Name and Surname
    nameC=Francesco Ferlin, % Author Name and Surname
    course=Computer Science and Engineering, % Study Programme (in Italian)
    academicyear={2023{-}24},  % Academic Year
} % These info will be put into your Title page 

% Define deliverable specific info
% Replace cell contents where neededs
\renewcommand{\headrulewidth}{0pt} % removing the horizontal line in the header
\begin{table}[h!]
    \begin{tabu} to \textwidth { X[0.3,r,p] X[0.7,l,p] }
        \hline

        \textbf{Deliverable:}   & RASD                                                                                           \\
        \textbf{Title:}         & Requirement Analysis and Verification Document                                                 \\
        \textbf{Authors:}       & Giacomo Orsenigo, Federico Saccani, Francesco Ferlin                                           \\
        \textbf{Version:}       & 0.1                                                                                            \\
        \textbf{Date:}          & 21{-}11{-}2023                                                                                 \\
        \textbf{Download page:} & https://github.com/Furrrlo/FerlinOrsenigoSaccani/                                              \\
        \textbf{Copyright:}     & Copyright © 2023, Giacomo Orsenigo, Federico Saccani, Francesco Ferlin {-} All rights reserved \\
        \hline
    \end{tabu}
\end{table}

%----------------------------------------------------------------------------
%	PREAMBLE PAGES: ABSTRACT (inglese e italiano), EXECUTIVE SUMMARY
%----------------------------------------------------------------------------
\startpreamble
\setcounter{page}{1} % Set page counter to 1

%----------------------------------------------------------------------------
%	LIST OF CONTENTS/FIGURES/TABLES/SYMBOLS
%----------------------------------------------------------------------------

% TABLE OF CONTENTS
\thispagestyle{empty}
\tableofcontents % Table of contents 
\thispagestyle{empty}
\cleardoublepage

%-------------------------------------------------------------------------
%	THESIS MAIN TEXT
%-------------------------------------------------------------------------
% In the main text of your thesis you can write the chapters in two different ways:
%
%(1) As presented in this template you can write:
%    \chapter{Title of the chapter}
%    *body of the chapter*
%
%(2) You can write your chapter in a separated .tex file and then include it in the main file with the following command:
%    \chapter{Title of the chapter}
%    \input{chapter_file.tex}
%
% Especially for long thesis, we recommend you the second option.

\addtocontents{toc}{\vspace{2em}} % Add a gap in the Contents, for aesthetics
\mainmatter % Begin numeric (1,2,3...) page numbering

% --------------------------------------------------------------------------
% NUMBERED CHAPTERS % Regular chapters following
% --------------------------------------------------------------------------

\chapter{Introduction}
% !TeX root = ../dd.tex
\comment{Explain what a repo slug is in definitions}


\chapter{Overall Description}
% !TeX root = ../rasd.tex
\section{Product perspective}
\subsection{Scenarios}

\begin{enumerate}
	\item \textbf{Tournament Creation by an Educator}\\
	      Luca Proserpio is an educator who works for a famous Italian university and is subscribed to the CKB Platform.\\
	      Luca wants to create a new tournament on the platform so that students can participate in it to improve their programming skills in Java.\\
	      Luca logs in, and the \emph{HOMEPAGE} will be shown.\\
	      Luca among the various options on the homepage decides to click on the \emph{"Create New Tournament"} button. \\
	      The platform shows the \emph{CREATE\_TOURNAMENT} page which contains:

	      \begin{enumerate}
		      \item Name field: to set the name of the tournament
		      \item Collaborators list: with all the educators in the platform
		      \item Deadline field: to define the deadline date to be able to register for the tournament
		      \item Create Tournament button: to proceed with the creation of the tournament
	      \end{enumerate}

	      Luca enters the name \emph{"Welcome Tournament School Year 2024"} in the Name field and sets the date 30/09/2023 in the Deadline field.\\
	      Next, he clicks on the list of educators selecting educators Gianpaolo Mariani, Mario Rossi, and Rosa Ballabio as he wants to allow them also to be able to create new
	      battles for the tournament he is creating.\\
	      Accordingly, he clicks on the Create Tournament button and the platform shows a message of successful tournament creation.
	      Luca will be shown the \emph{DETAIL\_TOURNAMENT} page of the newly created \emph{"Welcome Tournament School Year 2024"} for which, at the moment, as an empty list of
	      available
	      battles.
	      All students enrolled in the CKB Platform will be notified about the new creation of the tournament created by educator Luca.




	\item \textbf{Battle Creation by and Educator for an Existing Tournament}\\
	      Gianpaolo Mariani is an educator registered with the CKB Platform and wants to create a battle for the \emph{"Welcome Tournament School Year 2024"} tournament.\\
	      After logging in, the \emph{HOMEPAGE} shows the list of tournaments he has created or to which he has been added by other contributors. Gianpaolo clicks on the \emph{"Welcome
		      Tournament School Year 2024"} line found in the list of tournaments mentioned earlier.\\
	      After clicking, the platform shows the \emph{DETAIL\_TOURNAMENT} page of the selected tournament.
	      On the \emph{DETAIL\_TOURNAMENT} page, Gianpaolo clicks on the \emph{"Create New Battle"} button and the \emph{CREATE\_BATTLE} page is shown to him.\\
	      The \emph{CREATE\_BATTLE} page contains:

	      \begin{enumerate}
		      \item Kata Code Upload Section
		            \begin{enumerate}
			            \item Battle Description field
			            \item Form to upload test cases
			            \item Form to upload build automation scripts
		            \end{enumerate}
		      \item Field group policy: to set minimum and maximum number of students per group
		      \item Field registration deadline: to define maximum date for which students can register for the battle
		      \item Field final submission deadline: to define maximum date for which students can submit code to be evaluated
		      \item Scoring Configuration section

		            \begin{enumerate}
			            \item Form functional aspects
			            \item Form timeliness
			            \item Form quality level
			            \item Form optional manual evaluation
		            \end{enumerate}


		      \item Create Battle button
	      \end{enumerate}

	      Gianpaolo fills in all the fields by entering the description of the battle, the registration deadline as 6/11/2023, the final submission deadline as 23/12/2023, the minimum
	      and maximum number of students as 3, and does not edit any of the default entries in the Scoring Configuration section as he plans to edit it later before the battle is
	      actually started.\\
	      After that, he clicks on the Create Battle button and the platform redirect Gianpaolo to the \emph{DETAIL\_TOURNAMENT} page containing the new created battle in the list of available battles.
	      All students who had signed up for the tournament will receive a notification of the newly added battle created by educator Gianpaolo.



	\item \textbf{Student joins to an existing Tournament by receiving a notification}\\
	      Marco is an University student and thus is registered on the CKB Platform (ASSUMPTION). He wants to sign up for a tournament to improve his programming skills in Java as he notices that he has difficulty writing code using Object Oriented Programming. In the afternoon Marco receives a notification of a new tournament creation called \emph{"The Basics of Object-Oriented Programming in Java"}, he decides to take the opportunity and sign up.\\
	      Therefore, he accesses the platform by logging in and the platform redirect him to the \emph{HOMEPAGE}. Marco clicks on the record \emph{"The Basics of Object-Oriented Programming in Java"} listed in the list of available Tournaments for which he is not subscribed.\\
	      The platform shows Marco the \emph{SUBSCRIPTION\_TO\_A\_TOURNAMENT} page.\\
	      The \emph{SUBSCRIPTION\_TO\_A\_TOURNAMENT}  page shows information about the name and the description of the selected tournament. \\
	      Marco clicks on the Subscribe button, and the platform enrolls Marco in the tournament showing a confirmation registration message.

	\item \textbf{Students create a team for a tournament battle}\\
	      Marco, Stefano, and Carlo are students enrolled in \emph{"Welcome Tournament School Year 2024"} tournament.\\
	      They want to participate in one of the available tournament battles by creating a team.
	      Specifically, Marco logs in and in the \emph{HOMEPAGE} clicks on the tournament from the list of tournaments in which he is registered.\\
	      Marco is shown the \emph{DETAIL\_TOURNAMENT} page and within it he clicks on the first available battle.\\
	      He is shown the \emph{DETAIL\_BATTLE} page and clicks on the \emph{"Participate by creating a team"} button.\\
	      Marco already knows Stefano and decides to invite him to join his team, Stefano receives the a notification of the invitation and
	      clicks it to accept right away.\\
	      Because the tournament has a minimum requirement of 3 people, Marco leaves the page, in hopes to find the missing element to partecipate.\\
	      At a later time, Marco comes back on the page and invites Carlo, which he has just met during a lecture.\\
	      Carlo misses the notification, so he logs in and in the \emph{HOMEPAGE} finds the invitation. He clicks on the
	      \emph{accept} button.\\
	      Once Carlo has accepted, Marco goes back to the \emph{DETAIL\_TOURNAMENT} page, where he can see the team he has formed,
	      and clicks on the \emph{"Participate with the current team"} button.

	\item \textbf{Platform creates the GitHub repository when a battle's registration deadline expires}\\
	      Gianpaolo Mariano is an educator on the platform and had created a battle within the \emph{"Welcome Tournament School Year 2024"} tournament context with registration deadline 6/11/2023.\\
	      The system, when the registration deadline of the battle expires, creates the GitHub repository containing the code kata specified by Gianpaolo for the battle and sends the link of the newly created repository to all students who have registered in teams before the registration deadline expired.


	\item \textbf{Calculation of a team battle and personal students score}\\
	      Marco is a student and is participating in a battle with the team consisting of Stefano and Carlo.\\
	      Marco edits the project code and commits and pushes to the forked GitHub repository. \\
	      The platform fetches the code, analyzes it, and runs the prepared tests (test cases) for the battle in which the team is participating.\\
	      The platform updates the team's battle score by considering:
	      \begin{enumerate}
		      \item The number of successfully passed test cases
		      \item The number of days that have passed since the end of the registration deadline
		      \item The quality level of the sources extracted from the static analysis tools
	      \end{enumerate}
	      The platform also modifies each student's personal tournament score by summing all the battle scores received in that tournament.\\
	      Therefore, after the update, Marco,Stefano and Carlo go to the \emph{DETAIL\_BATTLE} page and view the new battle and tournament score assigned to them automatically by the platform.\\





	\item \textbf{Begin of Consolidation Stage when submission deadline of a battle expires}\\
	      Gianpaolo Mariano is a platform educator and had created a battle for the \emph{"Welcome Tournament School Year 2024"} with final submission deadline as 23/12/2023.\\
	      The platform at the end of the final submission deadline declares the start of the consolidation stage due to the fact that Gianpaolo had selected the \emph{"manual evaluation"} option for the battle he created.\\
	      Gianpaolo proceeds by hand analyzing the code submitted by the various groups and, in particular, decides to increase the score by +2 obtained by the group of students composed of Marco, Stefano and Carlo.\\
	      Afterwards, Gianpaolo using the platform declares the consolidation stage closed.\\
	      Marco, Stefano and Carlo and all other students who participated in the tournament battle receive a notification with the final battle rank.


	\item \textbf{Closing a tournament}\\
	      Luca Proserpio is a platform educator who had created the \emph{"Welcome Tournament School Year 2024"} tournament.\\
	      In March 2024, Luca decides to close the tournament.\\
	      All students enrolled in the CKB Platform will be notified about the closure of the Tournament.\\
	      Gianpaolo had also created the following gamification badges:
	      \begin{enumerate}
		      \item \emph{"Start2024_with_the_right_foot"}: awarded to all students enrolled in the tournament.
		      \item \emph{"Participant_2024"}: awarded to all students who have made at least 1 commit in any battle present in the tournament
	      \end{enumerate}
	      The platform analyzes all the details of the students who had registered for the tournament and accordingly assigns all students who met the requirements the relevant badges created by Luca.\\
	      In particular, Marco, Stefano, and Carlo had participated in several battles in the tournament and therefore receive both badges.\\
	      Samuele is another student who signed up for the tournament but never participated in any battle within the tournament, as a result the platform only assigns him the \emph{"Start2024_with_the_right_foot"} badge.

\end{enumerate}


\subsection{Domain class diagram}
\puml{puml/class-diagram}

\subsection{Statecharts}




\section{Product functions}



\section{User characteristics}
\subsection{Student}
The Student is a Client, i.e. a person who is able to access the CKB Platform.\\
They use the platform with the goal of honing their software development skills. Their participation takes place through several stages: they begin by signing up for tournaments and code kata battles, where they form or join teams of developers. Once involved, teams tackle a series of programming exercises in different languages. \\
Each student participates in tournaments within which he or she conducts battles and obtains scores.\\
They receive updates on battle scores, view final standings and personal tournament scores. Can also view and collect badges obtained during competitions.


\subsection{Educator}
The educator is a teacher who is able to access the CKB Platform. They use the platform with the goal of running tournaments and code kata battles, actively contributing to the students' educational process.\\
Its involvement follows several stages:\\
It begins by creating tournaments, defining the specifics of the battles, and establishing clear rules for evaluating students' proposed solutions with the ability to manually evaluate students' work.\\
An educator also manages gamification badges, i.e. rewards that provide additional incentive for students to succeed in their software development efforts.\\
His or her participation in the CKB Platform contributes to a challenging and competitive educational environment, encouraging students to reach higher levels of proficiency in software development.

\section{Assumptions, dependencies and constraints}
The following domain assumptions must hold in the world:

\begin{enumerate}[label=\textbf{D\arabic*}:,leftmargin=1.3cm]
	\item Students are enrolled in the school
	\item Educators teach in the school
	\item Students correctly fork the GitHub repository of the code kata battle
	\item Students correctly set up an automated workflow through GitHub Actions
	\item The grade assigned manually by the teachers reflects the work done
	\item The GitHub Action platform is working properly
	\item Test cases and build automation scripts uploaded by educators are correct
	\item The static analysis tool is working
\end{enumerate}

\chapter{Specific Requirements}
\section{External Interface Requirements}
\section{Functional Requirements}
\subsection{Use case diagrams}
\subsection{Use cases}
%Oppure facciamo tre sezioni separate per tabella, seq diagram e activity diagram,
% ma così mi sembra più leggibile.
\begin{enumerate}[label=\textbf{UC\arabic*}:,leftmargin=1.3cm]
    \item 
    \item 
    \item 
    \item 
    \item 
    \item 
    \item 
    \item \textbf{Closing a tournament}
          \begin{table}[H]
              %\caption*{\textbf{Title}}
              \centering
              \begin{tabular}{|l|p{11.9cm}|}
                  \hline
                  \textbf{Name}            & Closing a tournament                                 \\\hline
                  \textbf{Actor}           & Educator, Students                                   \\\hline
                  \textbf{Entry condition} &
                  \begin{itemize}
                      \item Educator has logged in
                      \item Educator has permission to modify the tournament
                      \item The tournament is in the consolidation stage
                  \end{itemize}                           \\\hline
                  \textbf{Event flow}      &
                  \begin{enumerate}
                      \item Educator click the \emph{close} button in the tournament page.
                      \item The platform calculate the final tournament rank.
                      \item The platform notifies students who participated in the tournament.
                      \item The platform assign badges to the student who fulfilled the rules.
                  \end{enumerate}         \\\hline
                  \textbf{Exit condition}  & The tournament is closed and the bedges are assigned \\\hline
                  %\textbf{Exception}       &                                                      \\\hline
                  %\textbf{Special reqs}    &                                                      \\\hline
              \end{tabular}
              \caption{Closing a tournament use case.}
              \label{table:Closing a tournament use case}
          \end{table}

          \puml{puml/UC8}

\end{enumerate}
\subsection{Requirements mapping}
\section{Performance Requirements}
\section{Design Constraints}
\section{Software System Attributes}
\subsection{Reliability}
The CKB platform does not manage critical operations.
If an operation fails, it can be re-executed without any particular consequences.
For example, if the evaluation of an exercise fails, students can resubmit their code.
It is therefore reasonable to have a failure rate around 1\%.

\subsection{Availability}
The platform should have the lowest downtime possible during the day.
A few hours of downtime are allowed between 1am and 5am, when student are usually not studying.
The platform shall therefore guarantee 99\% (two-nines) of availability, so that 3.65 days of downtime per year are allowed.

\subsection{Security}
Communication between the user and the CBK platform is encrypted with HTTPS. %Forse non posso dirlo qua(?)
Furthermore, users must only be able to perform operations that they are authorized to do.
For example, a student must not be able to change his grade.

\subsection{Maintainability}
The system should be divided in scalable and reusable modules,
which will be easier to maintain and replace in case of failure.
Ordinary maintenance, for bug fixes and improvements, will be scheduled during night time, when the user traffic is minimal.

\subsection{Portability}
The CKD platform does not require any particular hardware or software.
It must be accessible from any operating system with a modern web browser.
A mobile application that allows users to see the state of the battles can also be developed.
Since the mobile app does not require any special function, a non-native approach can be adopted.
It is therefore possible to use cross-platform development tools, which can speed up the development process.

\chapter{Formal analysis using alloy}
\lstinputlisting[language=alloy]{alloy/model.als}

\begin{adjustbox}{max size={\textwidth}{\textheight}}
    \includegraphics[]{alloy/diagrammone.pdf}
\end{adjustbox}

\chapter{Effort spent}
% !TeX root = ../rasd.tex
\begin{table}[H]
    %\caption*{\textbf{Title}}
    \centering
    \begin{tabular}{|l|l|l|}
        \hline
        \textbf{Member of group }                  & \textbf{Chapter}                          & \textbf{Time spent} \\\hline
        \multirow{5}{*}{\textbf{Giacomo Orsenigo}} & Introduction                              &                     \\
                                                   & Architectural Design                      & 5h                  \\
                                                   & User Interface Design                     &                     \\
                                                   & Requirements Traceability                 &                     \\
                                                   & Implementation, Integration and Test Plan &                     \\\hline
        \multirow{5}{*}{\textbf{Francesco Ferlin}} & Introduction                              & 1h                  \\
                                                   & Architectural Design                      & 10.5h               \\
                                                   & User Interface Design                     &                     \\
                                                   & Requirements Traceability                 & 1h                  \\
                                                   & Implementation, Integration and Test Plan & 2h                  \\\hline
        \multirow{5}{*}{\textbf{Federico Saccani}} & Introduction                              & 1.5h                    \\
                                                   & Architectural Design                      & 5h                  \\
                                                   & User Interface Design                     &                     \\
                                                   & Requirements Traceability                 &                     \\
                                                   & Implementation, Integration and Test Plan & 4h                    \\\hline
    \end{tabular}
    \caption{Time spent by each member of group.}
    \label{table:Time spent}
\end{table}


\chapter{References}
% !TeX root = ../rasd.tex
\comment{qui o nella bibliografia? e soprattutto che ci scrivo ooooo}
\begin{enumerate}
    \item \url{https://git-scm.com/}
    \item \url{https://git-scm.com/book/en/v2/Appendix-B%3A-Embedding-Git-in-your-Applications-Command-line-Git}
    \item \url{https://git-scm.com/book/en/v2/Appendix-B%3A-Embedding-Git-in-your-Applications-Libgit2}
    \item \url{https://git-scm.com/book/it/v2/Appendice-B%3A-Embedding-Git-in-your-Applications-JGit}
    \item \url{https://git-scm.com/book/en/v2/Appendix-B%3A-Embedding-Git-in-your-Applications-go-git}
    \item \url{https://www.eclipse.org/jgit/}
    \item \url{https://docs.github.com/en/authentication/connecting-to-github-with-ssh}
    \item \url{https://docs.github.com/en/graphql}
    \item \url{https://docs.github.com/en/rest?apiVersion=2022-11-28}
    \item \url{https://www.sonarsource.com/products/sonarqube/}
    \item \url{https://docs.sonarsource.com/sonarqube/latest/extension-guide/web-api/}
    \item \url{https://en.wikipedia.org/wiki/List_of_tools_for_static_code_analysis}
    \item \url{https://github.com/analysis-tools-dev/static-analysis}
\end{enumerate}

\chapter*{Introduction}

This document is intended to be both an example of the Polimi \LaTeX{} template for Master Theses,
as well as a short introduction to its use. It is not intended to be a general introduction to \LaTeX{} itself,
and the reader is assumed to be familiar with the basics of creating and compiling \LaTeX{} documents (see \cite{oetiker1995not, kottwitz2015latex}).
\\
The cover page of the thesis must contain all the relevant information:
title of the thesis, name of the Study Programme and School, name of the author,
student ID number, name of the supervisor, name(s) of the co-supervisor(s) (if any), academic year.
The above information are provided by filling all the entries in the command \verb|\puttitle{}|
in the title page section of this template.
\\
Be sure to select a title that is meaningful.
It should contain important keywords to be identified by indexer.
Keep the title as concise as possible and comprehensible even to people who are not experts in your field.
The title has to be chosen at the end of your work so that it accurately captures the main subject of the manuscript.
\\
Since a thesis might be a substantial document, it is convenient to break it into chapters.
You can create a new chapter as done in this template by simply using the following command
\begin{verbatim}
\chapter{Title of the chapter}
\end{verbatim}
followed by the body text.
\\
Especially for long manuscripts, it is recommended to give each chapter its own file.
In this case, you write your chapter in a separated \verb|chapter_n.tex| file
and then include it in the main file with the following command
\begin{verbatim}
\input{chapter_n.tex}
\end{verbatim}
It is recommended to give a label to each chapter by using the command
\begin{verbatim}
\label{ch:chapter_name}%
\end{verbatim}
where the argument is just a text string that you'll use to reference that part
as follows: \textit{Chapter~\ref{ch:chapter_one} contains \sc{an introduction to}  \dots}.\\
If necessary, an unnumbered chapter can be created by
\begin{verbatim}
\chapter*{Title of the unnumbered chapter}
\end{verbatim}

\chapter{Chapter one}
\label{ch:chapter_one}%
% The \label{...}% enables to remove the small indentation that is generated, always leave the % symbol.

In this chapter additional useful information are reported.

\section{Sections and subsections}
\label{sec:section_name}
Chapters are typically subdivided into sections and subsections, and, optionally,
subsubsections, paragraphs and subparagraphs.
All can have a title, but only sections and subsections are numbered.
A new section is created by the command
\begin{verbatim}
\section{Title of the section}
\end{verbatim}
The numbering can be turned off by using \verb|\section*{}|.
\\
A new subsection is created by the command
\begin{verbatim}
\subsection{Title of the subsection}
\end{verbatim}
and, similarly, the numbering can be turned off by adding an asterisk as follows
\begin{verbatim}
\subsection*{}
\end{verbatim}

\section{Equations}
\label{sec:eqs}
This section gives some examples of writing mathematical equations in your thesis.

Maxwell's equations read:
\begin{subequations}
    \label{eq:maxwell}
    \begin{align}[left=\empheqlbrace]
        \nabla\cdot \bm{D}                                         & = \rho, \label{eq:maxwell1}   \\
        \nabla \times \bm{E} +  \frac{\partial \bm{B}}{\partial t} & = \bm{0}, \label{eq:maxwell2} \\
        \nabla\cdot \bm{B}                                         & = 0, \label{eq:maxwell3}      \\
        \nabla \times \bm{H} - \frac{\partial \bm{D}}{\partial t}  & = \bm{J}. \label{eq:maxwell4}
    \end{align}
\end{subequations}

Equation~\eqref{eq:maxwell} is automatically labeled by \texttt{cleveref},
as well as Equation~\eqref{eq:maxwell1} and Equation~\eqref{eq:maxwell3}.
Thanks to the \verb|cleveref| package, there is no need to use \verb|\eqref|.
Remember that Equations have to be numbered only if they are referenced in the text.

Equations~\eqref{eq:maxwell_multilabels1}, \eqref{eq:maxwell_multilabels2}, \eqref{eq:maxwell_multilabels3}, and \eqref{eq:maxwell_multilabels4} show again Maxwell's equations without brace:
\begin{align}
    \nabla\cdot \bm{D}                                         & = \rho, \label{eq:maxwell_multilabels1}   \\
    \nabla \times \bm{E} +  \frac{\partial \bm{B}}{\partial t} & = \bm{0}, \label{eq:maxwell_multilabels2} \\
    \nabla\cdot \bm{B}                                         & = 0, \label{eq:maxwell_multilabels3}      \\
    \nabla \times \bm{H} - \frac{\partial \bm{D}}{\partial t}  & = \bm{J} \label{eq:maxwell_multilabels4}.
\end{align}

Equation~\eqref{eq:maxwell_singlelabel} is the same as before,
but with just one label:
\begin{equation}
    \label{eq:maxwell_singlelabel}
    \left\{
    \begin{aligned}
        \nabla\cdot \bm{D}                                         & = \rho,   \\
        \nabla \times \bm{E} +  \frac{\partial \bm{B}}{\partial t} & = \bm{0}, \\
        \nabla\cdot \bm{B}                                         & = 0,      \\
        \nabla \times \bm{H} - \frac{\partial \bm{D}}{\partial t}  & = \bm{J}.
    \end{aligned}
    \right.
\end{equation}

\section{Figures, Tables and Algorithms}
Figures, Tables and Algorithms have to contain a Caption that describe their content, and have to be properly reffered in the text.

\subsection{Figures}
\label{subsec:figures}

For including pictures in your text you can use \texttt{TikZ} for high-quality hand-made figures,
or just include them as usual with the command
\begin{verbatim}
\includegraphics[options]{filename.xxx}
\end{verbatim}
Here xxx is the correct format, e.g. \verb|.png|, \verb|.jpg|, \verb|.eps|, \dots.

\begin{figure}[H]
    \centering
    \includegraphics[width=0.3\textwidth]{logo_polimi_scritta.eps}
    \caption{Caption of the Figure to appear in the List of Figures.}
    \label{fig:quadtree}
\end{figure}

Thanks to the \texttt{\textbackslash subfloat} command, a single figure, such as Figure~\ref{fig:quadtree},
can contain multiple sub-figures with their own caption and label, e.g. \color{black} Figure~\ref{fig:polimi_logo1} and Figure~\ref{fig:polimi_logo2}.

\begin{figure}[H]
    \centering
    \subfloat[One PoliMi logo.\label{fig:polimi_logo1}]{
        \includegraphics[scale=0.5]{images/logo_polimi_scritta.eps}
    }
    \quad
    \subfloat[Another one PoliMi logo.\label{fig:polimi_logo2}]{
        \includegraphics[scale=0.5]{images/logo_polimi_scritta2.eps}
    }
    \caption[Shorter caption]{This is a very long caption you don't want to appear in the List of Figures.}
    \label{fig:quadtree2}
\end{figure}


\subsection{Tables}
\label{subsec:tables}

Within the environments \texttt{table} and  \texttt{tabular} you can create very fancy tables as the one shown in Table~\ref{table:example}.
\begin{table}[H]
    \caption*{\textbf{Title of Table (optional)}}
    \centering
    \begin{tabular}{|p{3em} c c c |}
        \hline
        \rowcolor{bluepoli!40} % comment this line to remove the color
                       & \textbf{column 1} & \textbf{column 2} & \textbf{column 3} \T\B \\
        \hline \hline
        \textbf{row 1} & 1                 & 2                 & 3 \T\B                 \\
        \textbf{row 2} & $\alpha$          & $\beta$           & $\gamma$ \T\B          \\
        \textbf{row 3} & alpha             & beta              & gamma \B               \\
        \hline
    \end{tabular}
    \\[10pt]
    \caption{Caption of the Table to appear in the List of Tables.}
    \label{table:example}
\end{table}

You can also consider to highlight selected columns or rows in order to make tables more readable.
Moreover, with the use of \texttt{table*} and the option \texttt{bp} it is possible to align them at the bottom of the page. One example is presented in Table~\ref{table:exampleC}.

\begin{table}[H]
    \centering
    \begin{tabular}{|p{3em} | c | c | c | c | c | c|}
        \hline
        %    \rowcolor{bluepoli!40}
                      & \textbf{column1} & \textbf{column2} & \textbf{column3} & \textbf{column4} & \textbf{column5} & \textbf{column6} \T\B \\
        \hline \hline
        \textbf{row1} & 1                & 2                & 3                & 4                & 5                & 6 \T\B                \\
        \textbf{row2} & a                & b                & c                & d                & e                & f \T\B                \\
        \textbf{row3} & $\alpha$         & $\beta$          & $\gamma$         & $\delta$         & $\phi$           & $\omega$ \T\B         \\
        \textbf{row4} & alpha            & beta             & gamma            & delta            & phi              & omega \B              \\
        \hline
    \end{tabular}
    \\[10pt]
    \caption{Highlighting the columns}
    \label{table:exampleC}
\end{table}

\begin{table}[H]
    \centering
    \begin{tabular}{|p{3em} c c c c c c|}
        \hline
        %    \rowcolor{bluepoli!40}
                      & \textbf{column1} & \textbf{column2} & \textbf{column3} & \textbf{column4} & \textbf{column5} & \textbf{column6} \T\B \\
        \hline \hline
        \textbf{row1} & 1                & 2                & 3                & 4                & 5                & 6 \T\B                \\
        \hline
        \textbf{row2} & a                & b                & c                & d                & e                & f \T\B                \\
        \hline
        \textbf{row3} & $\alpha$         & $\beta$          & $\gamma$         & $\delta$         & $\phi$           & $\omega$ \T\B         \\
        \hline
        \textbf{row4} & alpha            & beta             & gamma            & delta            & phi              & omega \B              \\
        \hline
    \end{tabular}
    \\[10pt]
    \caption{Highlighting the rows}
    \label{table:exampleR}
\end{table}

\subsection{Algorithms}
\label{subsec:algorithms}

Pseudo-algorithms can be written in \LaTeX{} with the \texttt{algorithm} and \texttt{algorithmic} packages.
An example is shown in Algorithm~\ref{alg:var}.
\begin{algorithm}[H]
    \label{alg:example}
    \caption{Name of the Algorithm}
    \label{alg:var}
    \label{protocol1}
    \begin{algorithmic}[1]
        \STATE Initial instructions
        \FOR{$for-condition$}
        \STATE{Some instructions}
        \IF{$if-condition$}
        \STATE{Some other instructions}
        \ENDIF
        \ENDFOR
        \WHILE{$while-condition$}
        \STATE{Some further instructions}
        \ENDWHILE
        \STATE Final instructions
    \end{algorithmic}
\end{algorithm}

\vspace{5mm}

\section{Theorems, propositions and lists}

\subsection{Theorems}
Theorems have to be formatted as:
\begin{theorem}
    \label{a_theorem}
    Write here your theorem.
\end{theorem}
\textit{Proof.} If useful you can report here the proof.

\subsection{Propositions}
Propositions have to be formatted as:
\begin{proposition}
    Write here your proposition.
\end{proposition}

\subsection{Lists}
How to  insert itemized lists:
\begin{itemize}
    \item first item;
    \item second item.
\end{itemize}
How to insert numbered lists:
\begin{enumerate}
    \item first item;
    \item second item.
\end{enumerate}

\section{Use of copyrighted material}

Each student is responsible for obtaining copyright permissions, if necessary, to include published material in the thesis.
This applies typically to third-party material published by someone else.

\section{Plagiarism}

You have to be sure to respect the rules on Copyright and avoid an involuntary plagiarism.
It is allowed to take other persons' ideas only if the author and his original work are clearly mentioned.
As stated in the Code of Ethics and Conduct, Politecnico di Milano \textit{promotes the integrity of research,
    condemns manipulation and the infringement of intellectual property}, and gives opportunity to all those
who carry out research activities to have an adequate training on ethical conduct and integrity while doing research.
To be sure to respect the copyright rules, read the guides on Copyright legislation and citation styles available
at:
\begin{verbatim}
https://www.biblio.polimi.it/en/tools/courses-and-tutorials
\end{verbatim}
You can also attend the courses which are periodically organized on "Bibliographic citations and bibliography management".

\section{Bibliography and citations}
Your thesis must contain a suitable Bibliography which lists all the sources consulted on developing the work.
The list of references is placed at the end of the manuscript after the chapter containing the conclusions.
We suggest to use the BibTeX package and save the bibliographic references  in the file \verb|Thesis_bibliography.bib|.
This is indeed a database containing all the information about the references. To cite in your manuscript, use the \verb|\cite{}| command as follows:
\\
\textit{Here is how you cite bibliography entries: \cite{knuth74}, or multiple ones at once: \cite{knuth92,lamport94}}.
\\
The bibliography and list of references are generated automatically by running BibTeX \cite{bibtex}.

\chapter{Conclusions and future developments}
\label{ch:conclusions}%
A final chapter containing the main conclusions of your research/study
and possible future developments of your work have to be inserted in this chapter.

%-------------------------------------------------------------------------
%	BIBLIOGRAPHY
%-------------------------------------------------------------------------

\addtocontents{toc}{\vspace{2em}} % Add a gap in the Contents, for aesthetics
\bibliography{bibliography} % The references information are stored in the file named "Thesis_bibliography.bib"

%-------------------------------------------------------------------------
%	APPENDICES
%-------------------------------------------------------------------------

\cleardoublepage
\addtocontents{toc}{\vspace{2em}} % Add a gap in the Contents, for aesthetics
\appendix
\chapter{Appendix A}
If you need to include an appendix to support the research in your thesis, you can place it at the end of the manuscript.
An appendix contains supplementary material (figures, tables, data, codes, mathematical proofs, surveys, \dots)
which supplement the main results contained in the previous chapters.

\chapter{Appendix B}
It may be necessary to include another appendix to better organize the presentation of supplementary material.

% LIST OF FIGURES
\listoffigures
% LIST OF TABLES
\listoftables
% LIST OF SYMBOLS
% Write out the List of Symbols in this page
\chapter*{List of Symbols} % You have to include a chapter for your list of symbols (
\begin{table}[H]
    \centering
    \begin{tabular}{lll}
        \textbf{Variable} & \textbf{Description} & \textbf{SI unit} \\\hline\\[-9px]
        $\bm{u}$          & solid displacement   & m                \\[2px]
        $\bm{u}_f$        & fluid displacement   & m                \\[2px]
    \end{tabular}
\end{table}

% ACKNOWLEDGEMENTS
\chapter*{Acknowledgements}
Here you might want to acknowledge someone.

\cleardoublepage

\end{document}
